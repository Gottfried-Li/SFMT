\documentclass{amsart}
\usepackage{amssymb}
\usepackage{graphicx}

\def\bZ{{\mathbb Z}}
\def\bN{{\mathbb N}}
\def\bQ{{\mathbb Q}}
\def\bF{{\mathbb F}}
\def\F2{{\bF_2}}

\def\CalB{{\mathcal B}}
\def\CalK{{\mathcal K}}
\def\CalP{{\mathcal P}}

\def\id{{\mathrm id}}
\def\bx{{\mathbf x}}
\def\ba{{\mathbf a}}

\theoremstyle{definition}
\newtheorem{definition}{Definition}[section]
\newtheorem{remark}[definition]{Remark}
\newtheorem{theorem}[definition]{Theorem}
\newtheorem{proposition}[definition]{Proposition}
\newtheorem{lemma}[definition]{Lemma}
\newtheorem{corollary}[definition]{Corollary}
\newtheorem{assumption}[definition]{Assumption}


\begin{document}

\title[Simple and Fast MT]
{Simple and Fast Mersenne Twister: A new variant of Mersenne Twister
improved in its speed} 

\author{Mutsuo Saito}
\address{Department of Mathematics \\ Hiroshima University\\
Hiroshima 739-8526, JAPAN}
\email{saito@math.sci.hiroshima-u.ac.jp}

\author{Makoto Matsumoto}
\address{Department of Mathematics \\ Hiroshima University\\
Hiroshima 739-8526, JAPAN}
\email{m-mat@math.sci.hiroshima-u.ac.jp}

\date{\today}
%\subjclass{Primary xxxxx; Secondary xxxxx}

\keywords{Simple and Fast Mersenne Twister, SFMT}

\maketitle

%\begin{abstract}
Good randomness is most important thing of pseudo random number
generator, but speed is important thing, too.
Here we suggest a high speed PRNG fit for resent CPU
architectures.
Our target CPUs are new CPUs such as Intel Pentium 4 or AMD64 have
long pipelines and SIMD instructions.

And our speed up plans are;

First, make all things simple. The simpleness is the source of
speed. MT has little bit complicated internal state, PMT has
calculation dependency of its lung.

Second, use TGFSR. because GFSR is array base and 
purpose of long pipe line is good performance for array access.

Third, generate 128 bit random number and use it as 32 bit
random number and 64 bit random number. Because,
generate 128 bit once is good for SIMD instructions.
So the recursion is like this. where x is 128 bit vector and
A, B, C is 128 x 128 bit Matrix.
\[x_{n+1} = Ax_1 + Bx_m + Cx_{n-1}\]
\[A = \left( \begin{array}{cccc}
    A' & 0 & 0 & 0 \\
    0 & A' & 0 & 0 \\
    0 & 0 & A' & 0 \\
    0 & 0 & 0 & A' \end{array} \right) \]
\[B = \left( \begin{array}{cccc}
    B' & I & 0 & 0 \\
    0 & B' & I & 0 \\
    0 & 0 & B' & I \\
    0 & 0 & 0 & B' \end{array} \right) \]
\[C = \left( \begin{array}{cccc}
    C' & 0 & 0 & 0 \\
    I & C' & 0 & 0 \\
    0 & I & C' & 0 \\
    0 & 0 & I & C' \end{array} \right) \]

There is a little problem to calculate k-distributions of 32 and
64 bit random numbers, but it can be solve by using additonal 
internal state.

Fourth,  we provide fill\_array function that fills 
given length array with random numbers. This function is
just like burst mode of memory access. For GFSR based PRNG,
the method that one function call returns one random number
is bottle neck of performance.

Fifth, use large internal state. To use advantage of pipeline fully,
internal state array of GFSR shoule be large, as far as it is
shorter than CPU cache size.

Sixth, use SIMD instruction function provided by Intel C++ Compiler
and GCC. First we expected that compiler optimization will
generate SIMD instructions when we write 128 bit operation 
by C languaage, but not. So we use compiler provided function/macro
as a second best plan.

Using these plans, we get a new PRNG which is 2 to 3 times 
faster than MT.

Finally, we can show a new PRNG has good randomness same as MT.
%\end{abstract}

\end{document}

